% -----------------------------------------------------------------------------
% -----------------------------------------------------------------------------
%  Universidade Federal do Amazonas
%  Instituto de Computação
%  Ciência da Computação
%
%  Adaptado de pacote abntex2 http://code.google.com/p/abntex2/
%     por Roberto Cidade Fonseca.
% -----------------------------------------------------------------------------
% -----------------------------------------------------------------------------

\documentclass[
	% -- opções da classe memoir --
	12pt,				% tamanho da fonte
	openright,			% capítulos começam em pág ímpar (insere página vazia caso preciso)
	oneside,
	a4paper,				% tamanho do papel.
	% -- opções da classe abntex2 --
	%chapter=TITLE,		% títulos de capítulos convertidos em letras maiúsculas
	%section=TITLE,		% títulos de seções convertidos em letras maiúsculas
	%subsection=TITLE,		% títulos de subseções convertidos em letras maiúsculas
	%subsubsection=TITLE,	% títulos de subsubseções convertidos em letras maiúsculas
	% -- opções do pacote babel --
	english,				% idioma adicional para hifenização
	brazil				% o último idioma é o principal do documento
]{abntex2/abntex2} % Entre chaves vai o caminho para o arquivo .cls

% ---
% Pacotes básicos
% ---
\usepackage{bookman}				% Usa a fonte Bookman Old Style
\usepackage[T1]{fontenc}			% Selecao de codigos de fonte.
\usepackage[utf8]{inputenc}		% Codificacao do documento (conversão automática dos acentos)
\usepackage{color}				% Controle das cores
\usepackage{graphicx}			% Inclusão de gráficos
\usepackage{microtype} 			% para melhorias de justificação

\usepackage[brazilian,hyperpageref]{backref}	 % Paginas com as citações na bibl
\usepackage[alf]{abntex2cite}	% Citações padrão ABNT

% ---
% variáveis como título, etc.
% ---
\newcommand{\tituloTCC}{Classificação de Produtos através de Modelos de Linguagem}

\titulo{\tituloTCC{}}
\autor{Rúben Jozafá Silva Belém}
\local{Manaus - AM}
\data{Setembro de 2018}
\orientador[Orientador(a)]{Prof. Dr. Edleno Silva de Moura}
\instituicao{%
  Universidade Federal do Amazonas
  \par
  Instituto de Computação}
\curso{%
  Bacharelado em Ciência da Computação}
\tipotrabalho{Monografia}
% O preambulo deve conter o tipo do trabalho, o objetivo,
% o nome da instituição e a área de concentração
\preambulo{Monografia de Graduação apresentada ao Instituto de Computação da Universidade Federal do Amazonas como requisito parcial para a obtenção do grau de bacharel em Ciência da Computação.}

% Informações do PDF
\makeatletter
\hypersetup{
    	%pagebackref=true,
	pdftitle={\@title},
	pdfauthor={\@author},
    	pdfsubject={\imprimirpreambulo},
    pdfcreator={LaTeX with abnTeX2},
	pdfkeywords={abnt}{latex}{abntex}{abntex2}{trabalho acadêmico},
	bookmarksdepth=4
}
\makeatother

% ---
% Espaçamentos entre linhas e parágrafos
% ---

% O tamanho do parágrafo é dado por:
\setlength{\parindent}{1.3cm}

% Controle do espaçamento entre um parágrafo e outro:
%\setlength{\parskip}{0.2cm}  % tente também \onelineskip

%\setbeforesecskip{3em}
%\setbeforesubsecskip{3em}

% ---
% compila o indice
% ---
\makeindex

% ---------------------------------------------------
% INICIO DE DOCUMENTO
% ---------------------------------------------------

\begin{document}

\noindent

% Seleciona o idioma do documento (conforme pacotes do babel)
%\selectlanguage{english}
\selectlanguage{brazil}

% Retira espaço extra obsoleto entre as frases.
\frenchspacing

% ----------------------------------------------------------
% ELEMENTOS PRÉ-TEXTUAIS
% ----------------------------------------------------------
% \pretextual

% ---
% Capa
% ---
\imprimircapa
% ---

% ---
% Folha de rosto
% (o * indica que haverá a ficha bibliográfica)
% ---
% \imprimirfolhaderosto*
% ---

% ---
% Inserir folha de aprovação
% ---

% Isto é um exemplo de Folha de aprovação, elemento obrigatório da NBR
% 14724/2011 (seção 4.2.1.3). Você pode utilizar este modelo até a aprovação
% do trabalho. Após isso, substitua todo o conteúdo deste arquivo por uma
% imagem da página assinada pela banca com o comando abaixo:
%
% \includepdf{folhadeaprovacao_final.pdf}
%-----


% \begin{folhadeaprovacao}
% 	\parindent=0pt
% 	\setlength{\ABNTEXsignskip}{1.5cm}

% 	Monografia de Graduação sob o título \textit{\tituloTCC{}} apresentada por Rúben Jozafá Silva Belém e aceita pelo Instituto de Computação da Universidade Federal do Amazonas, sendo aprovada por todos os membros da banca examinadora abaixo especificada:

% 	\assinatura{\fontsize{12}{15}\selectfont Titulação e nome do(a) orientador(a) \\ \fontsize{11}{15}\selectfont \imprimirorientadorRotulo~ \\ {\fontsize{10}{12}\selectfont Departamento \par Universidade}}
% 	\vspace{1cm}
% 	\assinatura{\fontsize{12}{15}\selectfont Titulação e nome do(a) membro da banca examinadora \\ \fontsize{11}{15}\selectfont Co-orientador(a), se houver \\ {\fontsize{10}{12}\selectfont Departamento \par Universidade}}
% 	\vspace{1cm}
% 	\assinatura{Titulação e nome do membro da banca examinadora \\ {\fontsize{10}{12}\selectfont Departamento \par Universidade}}
% 	\vspace{1cm}
% 	\assinatura{Titulação e nome do membro da banca examinadora \\ {\fontsize{10}{12}\selectfont Departamento \par Universidade}}
% 	\vfill

% 	\begin{center}
% 		\fontsize{12}{15}\selectfont
% 		\vspace*{0.5cm}
% 		\imprimirlocal, data de aprovação (por extenso).
% 		\vspace*{1cm}
% 	\end{center}

% \end{folhadeaprovacao}



% ---
% Dedicatória
% ---


% \begin{dedicatoria}
%   \vspace*{\fill}
%   \noindent
%   \leftskip=5cm

%   Homenagem que o autor presta a uma ou mais pessoas.

%   \vspace{5cm}
% \end{dedicatoria}


% ---
% Agradecimentos
% ---


% \begin{agradecimentos}

% Agradecimentos dirigidos àqueles que contribuíram de maneira relevante à elaboração do trabalho, sejam eles pessoas ou mesmo organizações.

% \end{agradecimentos}

% % ---
% % Epígrafe
% % ---
% \begin{epigrafe}
%     \vspace*{\fill}
% 	\begin{flushright}
% 		\textit{Citação}

% 		Autor
% 	\end{flushright}\vspace{4cm}
% \end{epigrafe}



% ---
% RESUMOS
% ---


% % resumo em português
% \setlength{\absparsep}{18pt} % ajusta o espaçamento dos parágrafos do resumo
% \begin{resumo}

%  	O resumo deve apresentar de forma concisa os pontos relevantes de um texto, fornecendo uma visão rápida e clara do conteúdo e das conclusões do trabalho. O texto, redigido na forma impessoal do verbo, é constituído de uma sequência de frases concisas e objetivas e não de uma simples enumeração de tópicos, não ultrapassando 500 palavras, seguido, logo abaixo, das palavras representativas do conteúdo do trabalho, isto é, palavras-chave e/ou descritores. Por exemplo, deve-se evitar, na redação do resumo, o uso de fórmulas, equações, diagramas e símbolos, optando-se, quando necessário, pela transcrição na forma extensa, além de não incluir citações bibliográficas.

%  \textit{Palavras-chave}: Palavra-chave 1, Palavra-chave 2, Palavra-chave 3.

% \end{resumo}

% % resumo em inglês
% \begin{resumo}[Abstract]
%  \begin{otherlanguage*}{english}
%   This is the english abstract.

%   \vspace{\onelineskip}

%   \noindent
%   \textit{Keywords}: Keyword 1, Keyword 2, Keyword 3.
%  \end{otherlanguage*}
% \end{resumo}



% % ---
% % inserir lista de figuras
% % ---
% \pdfbookmark[0]{\listfigurename}{lof}
% \listoffigures*
% \cleardoublepage
% % ---

% % ---
% % inserir lista de tabelas
% % ---
% \pdfbookmark[0]{\listtablename}{lot}
% \listoftables*
% \cleardoublepage
% % ---

% % ---
% % inserir lista de abreviaturas e siglas
% % ---
% \begin{siglas}
%   \item[ABNT] Associação Brasileira de Normas Técnicas
%   \item[abnTeX] ABsurdas Normas para TeX
% \end{siglas}
% % ---

% % ---
% % inserir lista de símbolos
% % ---
% \begin{simbolos}
%   \item[$ \lambda $] Lambda
% \end{simbolos}
% % ---

% ---
% inserir o sumario
% ---
\pdfbookmark[0]{\contentsname}{toc}
\tableofcontents*
\cleardoublepage



\textual

\chapter{Introdução}

	\section{Contextualização}

	\section{Objetivos}

	\section{Organização do Trabalho}


% ---
% Capítulo 2
% ---
\chapter{Revisão Bibliográfica}

	Muitos trabalhos encontrados na literatura abordam o problema de classificação de produtos. Alguns dos métodos clássicos de aprendizado de máquina podem ser utilizados para esse problema, como o \emph{SVM - Support Vector Machine} (Máquina de Vetores de Suporte) \cite{Joachims:1998}, \emph{K-NN - K Nearest Neighbours} (K vizinhos mais próximos) \cite{Chakrabarti:2003}, e \emph{Naive Bayes} \cite{rish2001empirical}. No entanto, as conclusões desses trabalhos apontam que utilizar desses métodos não produz um resultado tão satisfatório. Portanto, verificou-se trabalhos especificamente aplicados à classificação de produtos.

	A \emph{Deep Categorization Network} \cite{conf/kdd/2016}, ou \emph{DeepCN}, uma rede profunda de ponta a ponta formada por múltiplas redes neurais recorrentes (RNNs) é alimentada com metadados provindos de produtos de comércio eletrônico. A rede possui camadas totalmente conectadas e uma camada \emph{softmax}. Para cada atributo de item como nome, marca ou fabricante há uma RNN dedicada. Elas geram vetores de valor real que caracterizam a semântica das palavras, descartando assim um processo de treinamento prévio como o \emph{word2vec} \cite{mikolov2013distributed}.

	Outra abordagem é a de uma Rede Profunda de Fusão de Níveis de Decisão \cite{ZahavyMKM16}, para classificação de produtos multimodais. Um produto multimodal no contexto desse trabalho é composto por texto e imagem, os quais são utilizados como entrada. A rede multimodal melhora a precisão \emph{Top-1\%} em uma base coletada do \emph{Walmart}. São utilizadas duas Redes Neurais Convolutivas (CNN), uma para texto e outra para imagens. A CNN de texto tem precisão maior que a de imagem quando há poucos produtos. Quando há muitos, a CNN de imagem tem melhor desempenho. Isso serviu de incentivo para combinar as saídas das duas CNNs em uma terceira rede neural de "policiamento", que aprende a escolher entre o resultado da CNN de imagem ou de texto.

	\pagebreak
	Uma vez que a abordagem deste trabalho é baseada em \emph{modelos de linguagem}, também foi realizada uma busca a respeito dessa forma de classificação. Um dos trabalhos mais recentes e que se destaca é um estudo sobre algoritmos de suavização em modelos de linguagem para categorização de itens de um comércio eletrônico \cite{ShenRMS12}. Esses algoritmos são úteis para ajustar o estimador de máxima verossimilhança. Em outras palavras, regulam alguns parâmetros do modelo para a que a escassez de dados não afete tanto o desempenho. Os modelos de linguagem geralmente designam probabilidades aos termos de uma consulta, com base em suas ocorrências nos documentos. No entanto, algumas palavras da consulta podem não existir em nenhum dos documentos. O algoritmo de suavização então determina quanto se deve descontar da massa de probabilidade das palavras da consulta que aparecem na coleção, para atribuir tal desconto às palavras ``estranhas''.

	Analisou-se o desempenho do modelo de linguagem com cinco algoritmos de suavização: Suavização de \emph{Laplace}, de \emph{Jelinek–Mercer}, de Distribuição de \emph{Dirichlet}, de Encolhimento, e por último, o Desconto Absoluto. Os testes foram realizados com uma base de dados do \emph{eBay} que possui uma hierarquia de categorias de 7 níveis, com 34 categorias no primeiro nível, e um total de 19000 categorias ``folha'' (que não são ``pai'' de nenhuma outra categoria), com mais de 18 milhões de produtos. Dos cinco algoritmos, o que teve melhores resultados foi o da Suavização de Distribuição de \emph{Dirichlet}. Então, passou-se a considerar somente este último algoritmo no resto do trabalho. Foram analisadas também 3 tipos de influência nos resultados: a do tamanho da base de treino, do tamanho dos documentos das categorias, e da especificidade de palavras entre categorias (se as palavras entre duas categorias são pouco distintas, naturalmente a classificação se torna mais difícil).

% % ---
% % Capítulo 3
% % ---
% \chapter{Capítulo 3}

% 	Algumas regras devem ser observadas na redação da monografia:

% 	\begin{itemize}
% 		\item ser claro, preciso, direto, objetivo e conciso, utilizando frases curtas e evitando ordens inversas desnecessárias;

% 		\item construir períodos com no máximo duas ou três linhas, bem como parágrafos com cinco linhas cheias, em média, e no máximo oito (ou seja, não construir parágrafos e períodos muito longos, pois isso cansa o(s) leitor(es) e pode fazer com que ele(s) percam a linha de raciocínio desenvolvida);

% 		\item a simplicidade deve ser condição essencial do texto; a simplicidade do texto não implica necessariamente repetição de formas e frases desgastadas, uso exagerado de voz passiva (como será iniciado, será realizado), pobreza vocabular etc. Com palavras conhecidas de todos, é possível escrever de maneira original e criativa e produzir frases elegantes, variadas, fluentes e bem alinhavadas;

% 		\item adotar como norma a ordem direta, por ser aquela que conduz mais facilmente o leitor à essência do texto, dispensando detalhes irrelevantes e indo diretamente ao que interessa, sem rodeios (verborragias);

% 		\item não começar períodos ou parágrafos seguidos com a mesma palavra, nem usar repetidamente a mesma estrutura de frase;

% 		\item desprezar as longas descrições e relatar o fato no menor número possível de palavras;

% 		\item recorrer aos termos técnicos somente quando absolutamente indispensáveis e nesse caso colocar o seu significado entre parênteses (ou seja, não se deve admitir que todos os que lerão o trabalho já dispõem de algum conhecimento desenvolvido no mesmo);

% 		\item dispensar palavras e formas empoladas ou rebuscadas, que tentem
% transmitir ao leitor mera ideia de erudição;

% 		\item não perder de vista o universo vocabular do leitor, adotando a seguinte
% regra prática: nunca escrever o que não se diria;

% 		\item usar termos coloquiais ou de gíria com extrema parcimônia (ou mesmo
% nem serem utilizados) e apenas em casos muito especiais, para não darem ao leitor a ideia de vulgaridade e descaracterizar o trabalho;

% 		\item ser rigoroso na escolha das palavras do texto, desconfiando dos
% sinônimos perfeitos ou de termos que sirvam para todas as ocasiões;

% 		\item em geral, há uma palavra para definir uma situação;

% 		\item encadear o assunto de maneira suave e harmoniosa, evitando a
% criação de um texto onde os parágrafos se sucedem uns aos outros
% como compartimentos estanques, sem nenhuma fluência entre si;

% 		\item ter um extremo cuidado durante a redação do texto, principalmente
% com relação às regras gramaticais e ortográficas da língua;

% 		\item geralmente todo o texto é escrito na forma impessoal do verbo, não se utilizando,
% portanto, de termos em primeira pessoa, seja do plural ou do singular.

% 	\end{itemize}

% 	Continuação.

% 	\section{Seção 1}

% 		Teste de uma tabela:

% 		\begin{table}[htbp]
% 			\caption{Tabela sem sentido.}
% 			\label{tabela-ssentido}
% 			\begin{center}
% 			\begin{tabular}{|c|c|}
% 				\hline
% 				Título Coluna & Título Coluna \\
% 				1 & 2 \\
% 				\hline
% 				X & Y \\
% 				\hline
% 				X & W \\
% 				\hline
% 			\end{tabular}
% 			\end{center}
% 		\end{table}

% 	\section{Seção 2}

% 		Seção 2.

% 	\section{Seção 3}

% 		Seção 3.

% % - - -
% % Capítulo 4
% % - - -
% \chapter{Capítulo 4}

% 	\section{Seção 1}

% 		Teste de símbolo:

% 		$\lambda$

% 	\section{Seção 2}

% 		Teste de abreviaturas:

% % - - -
% % Capítulo 5
% % - - -
% \chapter{Capítulo 5}

% 	\section{Seção 1}

% 		Seção 1.

% 	\section{Seção 2}

% 		Alguns exemplos de citação:

% 		Na tese de Doutorado de Paquete \cite{paquete2007stochastic}, discute-se sobre algoritmos de busca local estocásticos aplicados a problemas de Otimização Combinatória considerando múltiplos objetivos. Por sua vez, o trabalho de \cite{knowles2003bounded}, publicado nos anais do IEEE CEC de 2003, mostra uma técnica de arquivamento também empregada no desenvolvimento de algoritmos evolucionários multiobjetivo, trabalho esse posteriormente estendido para um capítulo de livro dos mesmos autores \cite{gandibleux2004metaheuristics}. Por m, no relatório técnico de Jaszkiewicz (1998), fala-se sobre um algoritmo genético híbrido para problemas multi- critério, enquanto no artigo de jornal de Lopez et al. (LÓPEZ-IBÁÑEZ; PAQUETE; STÜTZLE, 2006) trata-se do trade-o entre algoritmos genéticos e metodologias de busca local, também aplicados no contexto multicritério e relacionado de alguma forma ao trabalho de Jaszkiewicz (1998).

% 		Outros exemplos relacionados encontram-se em (SILBERSCHATZ; KORTH; SUDARSHAN, 2002) (livro), (TURAU, 2001) (referência da Web) e (AGRA, 2004) (dissertação de Mestrado).

% 		\subsection{Subseção 2.1}

% 			Seção 2.1

% 	\section{Seção 3}

% 		Seção 3

% \chapter{Considerações finais}

% 	As considerações finais formam a parte final (fechamento) do texto, sendo dito de forma resumida (1) o que foi desenvolvido no presente trabalho e quais os resultados do mesmo, (2) o que se pôde concluir após o desenvolvimento bem como as principais contribuições do trabalho, e (3) perspectivas para o desenvolvimento de trabalhos futuros. O texto referente às considerações finais do autor deve salientar a extensão e os resultados da contribuição do trabalho e os argumentos utilizados estar baseados em dados comprovados e fundamentados nos resultados e na discussão do texto, contendo deduções lógicas correspondentes aos objetivos do trabalho, propostos inicialmente.


% ----------------------------------------------------------
% ELEMENTOS PÓS-TEXTUAIS
% ----------------------------------------------------------
\postextual
\Spacing{1.5}
% -----------------------------------------------------------------------------
% Referencias Bibliograficas
% -----------------------------------------------------------------------------
\bibliography{referencias}

% ----------------------------------------------------------
% Apêndices
% ----------------------------------------------------------

% ---
% Inicia os apêndices
% ---
% \begin{apendicesenv}

% % ----------------------------------------------------------
% \chapter{Primeiro apêndice}
% % ----------------------------------------------------------

% Os apêndices são textos ou documentos elaborados pelo autor, a fim de complementar sua argumentação, sem prejuízo da unidade nuclear do trabalho.

% \end{apendicesenv}

% % Anexos
% % ----------------------------------------------------------

% % ---
% % Inicia os anexos
% % ---
% \begin{anexosenv}

% % ---
% \chapter{Primeiro anexo.}
% % ---

% Os anexos são textos ou documentos não elaborados pelo autor, que servem de fundamentação, comprovação e ilustração.

% \end{anexosenv}

\end{document}